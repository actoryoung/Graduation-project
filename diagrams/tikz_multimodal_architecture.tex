% 多模态情感分析系统架构图
% 使用LaTeX TikZ绘制,适合论文发表
% 编译方式: pdflatex tikz_multimodal_architecture.tex

\documentclass[tikz,border=10pt]{standalone}
\usepackage{tikz}
\usepackage{amssymb}
\usepackage{xeCJK}
\setCJKmainfont{SimSun}
\usetikzlibrary{shapes.geometric, arrows.meta, positioning, fit, backgrounds, calc, shadows.blur, decorations.pathreplacing}

% --- 颜色定义 ---
% 文本模态 - 蓝色系
\definecolor{textFill}{HTML}{E3F2FD}
\definecolor{textStroke}{HTML}{1976D2}

% 音频模态 - 绿色系
\definecolor{audioFill}{HTML}{E8F5E9}
\definecolor{audioStroke}{HTML}{43A047}

% 视频模态 - 红色系
\definecolor{videoFill}{HTML}{FFEBEE}
\definecolor{videoStroke}{HTML}{E53935}

% 融合层 - 紫色系
\definecolor{fusionFill}{HTML}{F3E5F5}
\definecolor{fusionStroke}{HTML}{7B1FA2}

% 操作层 - 灰蓝色系
\definecolor{opFill}{HTML}{E1F5FE}
\definecolor{opStroke}{HTML}{0277BD}

% 分类层 - 橙色系
\definecolor{classFill}{HTML}{FFF3E0}
\definecolor{classStroke}{HTML}{FF9800}

\begin{document}

\begin{tikzpicture}[
    node distance=1.0cm and 1.5cm,
    font=\sffamily\footnotesize,
    >=Stealth,
    % --- 样式定义 ---
    textNode/.style={
        rectangle, rounded corners=4pt,
        draw=textStroke, very thick,
        fill=textFill,
        minimum width=2.2cm, minimum height=1.0cm,
        align=center,
        drop shadow
    },
    audioNode/.style={
        rectangle, rounded corners=4pt,
        draw=audioStroke, very thick,
        fill=audioFill,
        minimum width=2.2cm, minimum height=1.0cm,
        align=center,
        drop shadow
    },
    videoNode/.style={
        rectangle, rounded corners=4pt,
        draw=videoStroke, very thick,
        fill=videoFill,
        minimum width=2.2cm, minimum height=1.0cm,
        align=center,
        drop shadow
    },
    opNode/.style={
        rectangle, rounded corners=3pt,
        draw=opStroke, thick,
        fill=opFill,
        minimum width=2.0cm, minimum height=0.9cm,
        align=center,
        drop shadow
    },
    fusionNode/.style={
        rectangle, rounded corners=4pt,
        draw=fusionStroke, very thick,
        fill=fusionFill,
        minimum width=2.5cm, minimum height=1.1cm,
        align=center,
        drop shadow
    },
    classNode/.style={
        rectangle, rounded corners=4pt,
        draw=classStroke, very thick,
        fill=classFill,
        minimum width=2.2cm, minimum height=1.0cm,
        align=center,
        drop shadow
    },
    group/.style={
        draw=gray!40, dashed, rounded corners=8pt,
        inner sep=10pt, fill=gray!5
    },
    edgeLabel/.style={
        font=\scriptsize\itshape,
        text=gray!70,
        align=center,
        midway
    }
]

    % ===== 输入层 =====
    \node[textNode] (textInput) {文本输入\\Text\\768D};
    \node[audioNode, below=of textInput] (audioInput) {音频输入\\Audio\\74D};
    \node[videoNode, below=of audioInput] (videoInput) {视频输入\\Video\\710D};

    % ===== 特征编码层 =====
    \node[opNode, right=of textInput] (textEnc) {文本编码器\\Linear(768→256)};
    \node[opNode, right=of audioInput] (audioEnc) {音频编码器\\Linear(74→128)};
    \node[opNode, right=of videoInput] (videoEnc) {视频编码器\\Linear(710→256)};

    % 音频投影层
    \node[opNode, right=of audioEnc] (audioProj) {音频投影\\Linear(128→256)};

    % ===== 位置编码层 =====
    \node[fusionNode, right=3.5cm of textEnc] (posEnc) {位置编码\\Positional\\Encoding};

    % 连接编码器到位置编码
    \draw[->, thick, textStroke] (textEnc) -- (posEnc) node[edgeLabel, above] {256D};
    \draw[->, thick, audioStroke] (audioProj) -- (posEnc.north -| posEnc.west) node[edgeLabel, above] {256D};
    \draw[->, thick, videoStroke] (videoEnc) -- (posEnc.south -| posEnc.west) node[edgeLabel, below] {256D};

    % ===== Transformer融合层 =====
    \node[fusionNode, right=of posEnc] (transFusion) {Transformer融合\\Multi-Head\\Attention\\4头×2层};

    % ===== 融合聚合层 =====
    \node[opNode, right=of transFusion] (fusionAgg) {聚合与融合\\Mean Pooling\\+ MLP};

    % ===== 分类器 =====
    \node[classNode, right=of fusionAgg] (classifier) {分类器\\Classifier\\Linear(128→7)};

    % ===== 输出层 =====
    \node[opNode, right=of classifier] (output) {情感预测\\7类输出\\+置信度};

    % ===== 主连接 =====
    \draw[->, very thick, gray!70] (textInput) -- (textEnc);
    \draw[->, very thick, gray!70] (audioInput) -- (audioEnc);
    \draw[->, very thick, gray!70] (videoInput) -- (videoEnc);

    \draw[->, very thick, gray!70] (audioEnc) -- (audioProj);

    \draw[->, very thick, fusionStroke] (posEnc) -- (transFusion);
    \draw[->, very thick, fusionStroke] (transFusion) -- (fusionAgg);
    \draw[->, very thick, classStroke] (fusionAgg) -- (classifier);
    \draw[->, very thick, classStroke] (classifier) -- (output);

    % ===== 功能分组 =====
    \begin{scope}[on background layer]
        % 输入组
        \node[group, fit=(textInput)(audioInput)(videoInput)] (inputGroup) {};
        \node[above=0.1cm of inputGroup.north, font=\scriptsize\bfseries, color=gray!60] {输入层};

        % 编码组
        \node[group, fit=(textEnc)(audioEnc)(audioProj)(videoEnc)] (encGroup) {};
        \node[above=0.1cm of encGroup.north, font=\scriptsize\bfseries, color=gray!60] {特征编码层};

        % 融合组
        \node[group, fit=(posEnc)(transFusion)(fusionAgg)] (fusionGroup) {};
        \node[above=0.1cm of fusionGroup.north, font=\scriptsize\bfseries, color=gray!60] {跨模态融合层};

        % 输出组
        \node[group, fit=(classifier)(output)] (outputGroup) {};
        \node[above=0.1cm of outputGroup.north, font=\scriptsize\bfseries, color=gray!60] {输出层};
    \end{scope}

    % ===== 图例 =====
    \node[anchor=south east, font=\tiny] at (current bounding box.north east) {
        \begin{tikzpicture}
            \node[rectangle, rounded corners=2pt, draw=textStroke, fill=textFill, minimum width=0.5cm, minimum height=0.3cm] at (0,0) {};
            \node[right=0.1cm] at (0.3,0) {文本};
            \node[rectangle, rounded corners=2pt, draw=audioStroke, fill=audioFill, minimum width=0.5cm, minimum height=0.3cm] at (1.2,0) {};
            \node[right=0.1cm] at (1.5,0) {音频};
            \node[rectangle, rounded corners=2pt, draw=videoStroke, fill=videoFill, minimum width=0.5cm, minimum height=0.3cm] at (2.4,0) {};
            \node[right=0.1cm] at (2.7,0) {视频};
        \end{tikzpicture}
    };

\end{tikzpicture}
\end{document}
